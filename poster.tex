%%%%%%%%%%%%%%%%%%%%%%%%%%%%%%%%%%%%%%%%%
% Jacobs Landscape Poster
% LaTeX Template
% Version 1.1 (14/06/14)
%
% Created by:
% Computational Physics and Biophysics Group, Jacobs University
% https://teamwork.jacobs-university.de:8443/confluence/display/CoPandBiG/LaTeX+Poster
% 
% Further modified by:
% Nathaniel Johnston (nathaniel@njohnston.ca)
%
% This template has been downloaded from:
% http://www.LaTeXTemplates.com
%
% License:
% CC BY-NC-SA 3.0 (http://creativecommons.org/licenses/by-nc-sa/3.0/)
%
%%%%%%%%%%%%%%%%%%%%%%%%%%%%%%%%%%%%%%%%%

%----------------------------------------------------------------------------------------
%	PACKAGES AND OTHER DOCUMENT CONFIGURATIONS
%----------------------------------------------------------------------------------------

\documentclass[final]{beamer}

\usepackage[latin1]{inputenc}
\usepackage[percent]{overpic}
\usepackage{tikz}
\usetikzlibrary{arrows, shapes}
% This is not an official TikZ library. Use at your own risk!

\makeatletter
% alternative latex arrow
\pgfarrowsdeclare{latexnew}{latexnew}
{
  \ifdim\pgfgetarrowoptions{latexnew}=-1pt%
    \pgfutil@tempdima=0.28pt%
    \pgfutil@tempdimb=\pgflinewidth%
    \ifdim\pgfinnerlinewidth>0pt%
      \pgfmathsetlength\pgfutil@tempdimb{.6\pgflinewidth-.4*\pgfinnerlinewidth}%
    \fi%
    \advance\pgfutil@tempdima by.3\pgfutil@tempdimb%
  \else%
    \pgfutil@tempdima=\pgfgetarrowoptions{latexnew}%
    \divide\pgfutil@tempdima by 10%
  \fi%
  \pgfarrowsleftextend{+-1\pgfutil@tempdima}%
  \pgfarrowsrightextend{+9\pgfutil@tempdima}%
}
{
  \ifdim\pgfgetarrowoptions{latexnew}=-1pt%
    \pgfutil@tempdima=0.28pt%
    \pgfutil@tempdimb=\pgflinewidth%
    \ifdim\pgfinnerlinewidth>0pt%
      \pgfmathsetlength\pgfutil@tempdimb{.6\pgflinewidth-.4*\pgfinnerlinewidth}%
    \fi%
    \advance\pgfutil@tempdima by.3\pgfutil@tempdimb%
  \else%
    \pgfutil@tempdima=\pgfgetarrowoptions{latexnew}%
    \divide\pgfutil@tempdima by 10%
    \pgfsetlinewidth{0bp}%
  \fi%
  \pgfpathmoveto{\pgfqpoint{9\pgfutil@tempdima}{0pt}}
  \pgfpathcurveto
  {\pgfqpoint{6.3333\pgfutil@tempdima}{.5\pgfutil@tempdima}}
  {\pgfqpoint{2\pgfutil@tempdima}{2\pgfutil@tempdima}}
  {\pgfqpoint{-1\pgfutil@tempdima}{3.75\pgfutil@tempdima}}
  \pgfpathlineto{\pgfqpoint{-1\pgfutil@tempdima}{-3.75\pgfutil@tempdima}}
  \pgfpathcurveto
  {\pgfqpoint{2\pgfutil@tempdima}{-2\pgfutil@tempdima}}
  {\pgfqpoint{6.3333\pgfutil@tempdima}{-.5\pgfutil@tempdima}}
  {\pgfqpoint{9\pgfutil@tempdima}{0pt}}
  \pgfusepathqfill
}

% alternative latex reversed arrow
\pgfarrowsdeclarereversed{latexnew reversed}{latexnew reversed}{latexnew}{latexnew}

% alternative latex' arrow
\pgfarrowsdeclare{latex'new}{latex'new}
{
  \ifdim\pgfgetarrowoptions{latex'new}=-1pt%
    \pgfutil@tempdima=0.28pt%
    \advance\pgfutil@tempdima by.3\pgflinewidth%
  \else%
    \pgfutil@tempdima=\pgfgetarrowoptions{latex'new}%
    \divide\pgfutil@tempdima by 10%
  \fi%
  \pgfarrowsleftextend{+-4\pgfutil@tempdima}
  \pgfarrowsrightextend{+6\pgfutil@tempdima}
}
{
  \ifdim\pgfgetarrowoptions{latex'new}=-1pt%
    \pgfutil@tempdima=0.28pt%
    \advance\pgfutil@tempdima by.3\pgflinewidth%
  \else%
    \pgfutil@tempdima=\pgfgetarrowoptions{latex'new}%
    \divide\pgfutil@tempdima by 10%
    \pgfsetlinewidth{0bp}%
  \fi%
  \pgfpathmoveto{\pgfqpoint{6\pgfutil@tempdima}{0\pgfutil@tempdima}}
  \pgfpathcurveto
  {\pgfqpoint{3.5\pgfutil@tempdima}{.5\pgfutil@tempdima}}
  {\pgfqpoint{-1\pgfutil@tempdima}{1.5\pgfutil@tempdima}}
  {\pgfqpoint{-4\pgfutil@tempdima}{3.75\pgfutil@tempdima}}
  \pgfpathcurveto
  {\pgfqpoint{-1.5\pgfutil@tempdima}{1\pgfutil@tempdima}}
  {\pgfqpoint{-1.5\pgfutil@tempdima}{-1\pgfutil@tempdima}}
  {\pgfqpoint{-4\pgfutil@tempdima}{-3.75\pgfutil@tempdima}}
  \pgfpathcurveto
  {\pgfqpoint{-1\pgfutil@tempdima}{-1.5\pgfutil@tempdima}}
  {\pgfqpoint{3.5\pgfutil@tempdima}{-.5\pgfutil@tempdima}}
  {\pgfqpoint{6\pgfutil@tempdima}{0\pgfutil@tempdima}}
  \pgfusepathqfill
}

% alternative latex' reversed arrow
\pgfarrowsdeclarereversed{latex'new reversed}{latex'new reversed}{latex'new}{latex'new}

% alternative o arrow
\pgfarrowsdeclare{onew}{onew}
{
  \pgfarrowsleftextend{+-.5\pgflinewidth}
  \ifdim\pgfgetarrowoptions{onew}=-1pt%
    \pgfutil@tempdima=0.4pt%
    \advance\pgfutil@tempdima by.2\pgflinewidth%
    \pgfutil@tempdimb=9\pgfutil@tempdima\advance\pgfutil@tempdimb by.5\pgflinewidth%
    \pgfarrowsrightextend{+\pgfutil@tempdimb}%
  \else%
    \pgfutil@tempdima=\pgfgetarrowoptions{onew}%
    \advance\pgfutil@tempdima by -0.5\pgflinewidth%
    \pgfarrowsrightextend{+\pgfutil@tempdima}%
  \fi%
}
{ 
  \ifdim\pgfgetarrowoptions{onew}=-1pt%
    \pgfutil@tempdima=0.4pt%
    \advance\pgfutil@tempdima by.2\pgflinewidth%
    \pgfutil@tempdimb=0pt%
  \else%
    \pgfutil@tempdima=\pgfgetarrowoptions{onew}%
    \divide\pgfutil@tempdima by 9%
    \pgfutil@tempdimb=0.5\pgflinewidth%
  \fi%
  \pgfsetdash{}{+0pt}
  \pgfpathcircle{\pgfpointadd{\pgfqpoint{4.5\pgfutil@tempdima}{0bp}}%
                             {\pgfqpoint{-\pgfutil@tempdimb}{0bp}}}%
                {4.5\pgfutil@tempdima-\pgfutil@tempdimb}%
  \pgfusepathqstroke
}

% alternative square arrow
\pgfarrowsdeclare{squarenew}{squarenew}
{
 \ifdim\pgfgetarrowoptions{squarenew}=-1pt%
   \pgfutil@tempdima=0.4pt
   \advance\pgfutil@tempdima by.275\pgflinewidth%
   \pgfarrowsleftextend{+-\pgfutil@tempdima}
   \advance\pgfutil@tempdima by.5\pgflinewidth
   \pgfarrowsrightextend{+\pgfutil@tempdima}
 \else%
   \pgfutil@tempdima=\pgfgetarrowoptions{squarenew}%
   \divide\pgfutil@tempdima by 8%
   \pgfarrowsleftextend{+-7\pgfutil@tempdima}%
   \pgfarrowsrightextend{+1\pgfutil@tempdima}%
 \fi%
}
{
 \ifdim\pgfgetarrowoptions{squarenew}=-1pt%
   \pgfutil@tempdima=0.4pt%
   \advance\pgfutil@tempdima by.275\pgflinewidth%
   \pgfutil@tempdimb=0pt%
 \else%
   \pgfutil@tempdima=\pgfgetarrowoptions{squarenew}%   
   \divide\pgfutil@tempdima by 8%
   \pgfutil@tempdimb=0.5\pgflinewidth%
 \fi%
 \pgfsetdash{}{+0pt}
 \pgfsetroundjoin
 \pgfpathmoveto{\pgfpointadd{\pgfqpoint{1\pgfutil@tempdima}{4\pgfutil@tempdima}}
                            {\pgfqpoint{-\pgfutil@tempdimb}{-\pgfutil@tempdimb}}}
 \pgfpathlineto{\pgfpointadd{\pgfqpoint{-7\pgfutil@tempdima}{4\pgfutil@tempdima}}
                            {\pgfqpoint{\pgfutil@tempdimb}{-\pgfutil@tempdimb}}}
 \pgfpathlineto{\pgfpointadd{\pgfqpoint{-7\pgfutil@tempdima}{-4\pgfutil@tempdima}}
                            {\pgfqpoint{\pgfutil@tempdimb}{\pgfutil@tempdimb}}}
 \pgfpathlineto{\pgfpointadd{\pgfqpoint{1\pgfutil@tempdima}{-4\pgfutil@tempdima}}
                            {\pgfqpoint{-\pgfutil@tempdimb}{\pgfutil@tempdimb}}}
 \pgfpathclose
 \pgfusepathqfillstroke
}

% alternative stealth arrow
\pgfarrowsdeclare{stealthnew}{stealthnew}
{
  \ifdim\pgfgetarrowoptions{stealthnew}=-1pt%
    \pgfutil@tempdima=0.28pt%
    \pgfutil@tempdimb=\pgflinewidth%
    \ifdim\pgfinnerlinewidth>0pt%
      \pgfmathsetlength\pgfutil@tempdimb{.6\pgflinewidth-.4*\pgfinnerlinewidth}%
    \fi%
    \advance\pgfutil@tempdima by.3\pgfutil@tempdimb%
  \else%
    \pgfutil@tempdima=\pgfgetarrowoptions{stealthnew}%
    \divide\pgfutil@tempdima by 8%
  \fi%
  \pgfarrowsleftextend{+-3\pgfutil@tempdima}
  \pgfarrowsrightextend{+5\pgfutil@tempdima}
}
{
  \ifdim\pgfgetarrowoptions{stealthnew}=-1pt%
    \pgfutil@tempdima=0.28pt%
    \pgfutil@tempdimb=\pgflinewidth%
    \ifdim\pgfinnerlinewidth>0pt%
      \pgfmathsetlength\pgfutil@tempdimb{.6\pgflinewidth-.4*\pgfinnerlinewidth}%
    \fi%
    \advance\pgfutil@tempdima by.3\pgfutil@tempdimb%
  \else%
    \pgfutil@tempdima=\pgfgetarrowoptions{stealthnew}%
    \divide\pgfutil@tempdima by 8%
    \pgfsetlinewidth{0bp}%
  \fi%
  \pgfpathmoveto{\pgfqpoint{5\pgfutil@tempdima}{0pt}}
  \pgfpathlineto{\pgfqpoint{-3\pgfutil@tempdima}{4\pgfutil@tempdima}}
  \pgfpathlineto{\pgfpointorigin}
  \pgfpathlineto{\pgfqpoint{-3\pgfutil@tempdima}{-4\pgfutil@tempdima}}
  \pgfusepathqfill
}

% alternative stealth reversed arrow
\pgfarrowsdeclarereversed{stealthnew reversed}{stealthnew reversed}{stealthnew}{stealthnew}

% alternative to arrow
\pgfarrowsdeclare{tonew}{tonew}
{
  \ifdim\pgfgetarrowoptions{tonew}=-1pt%
    \pgfutil@tempdima=0.84pt%
    \advance\pgfutil@tempdima by1.3\pgflinewidth%
    \pgfutil@tempdimb=0.21pt%
    \advance\pgfutil@tempdimb by.625\pgflinewidth%
  \else%
    \pgfutil@tempdima=\pgfgetarrowoptions{tonew}%
    \pgfarrowsleftextend{+-0.8\pgfutil@tempdima}%
    \pgfarrowsrightextend{+0.2\pgfutil@tempdima}%
  \fi%
}
{
  \ifdim\pgfgetarrowoptions{tonew}=-1pt%
    \pgfutil@tempdima=0.28pt%
    \advance\pgfutil@tempdima by.3\pgflinewidth%
    \pgfutil@tempdimb=0pt,%
  \else%
    \pgfutil@tempdima=\pgfgetarrowoptions{tonew}%
    \multiply\pgfutil@tempdima by 100%
    \divide\pgfutil@tempdima by 375%
    \pgfutil@tempdimb=0.4\pgflinewidth%
  \fi%
  \pgfsetdash{}{+0pt}
  \pgfsetroundcap
  \pgfsetroundjoin
  \pgfpathmoveto{\pgfpointorigin}
  \pgflineto{\pgfpointadd{\pgfpoint{0.75\pgfutil@tempdima}{0bp}}
                         {\pgfqpoint{-2\pgfutil@tempdimb}{0bp}}}
  \pgfusepathqstroke
  \pgfsetlinewidth{0.8\pgflinewidth}
  \pgfpathmoveto{\pgfpointadd{\pgfqpoint{-3\pgfutil@tempdima}{4\pgfutil@tempdima}}
                             {\pgfqpoint{\pgfutil@tempdimb}{0bp}}}
  \pgfpathcurveto
  {\pgfpointadd{\pgfqpoint{-2.75\pgfutil@tempdima}{2.5\pgfutil@tempdima}}
               {\pgfqpoint{0.5\pgfutil@tempdimb}{0bp}}}
  {\pgfpointadd{\pgfqpoint{0pt}{0.25\pgfutil@tempdima}}
               {\pgfqpoint{-0.5\pgfutil@tempdimb}{0bp}}}
  {\pgfpointadd{\pgfqpoint{0.75\pgfutil@tempdima}{0pt}}
               {\pgfqpoint{-\pgfutil@tempdimb}{0bp}}}
  \pgfpathcurveto
  {\pgfpointadd{\pgfqpoint{0pt}{-0.25\pgfutil@tempdima}}
               {\pgfqpoint{-0.5\pgfutil@tempdimb}{0bp}}}
  {\pgfpointadd{\pgfqpoint{-2.75\pgfutil@tempdima}{-2.5\pgfutil@tempdima}}
               {\pgfqpoint{0.5\pgfutil@tempdimb}{0bp}}}
  {\pgfpointadd{\pgfqpoint{-3\pgfutil@tempdima}{-4\pgfutil@tempdima}}
               {\pgfqpoint{\pgfutil@tempdimb}{0bp}}}
  \pgfusepathqstroke
}

% alias alternative to arrow
\pgfarrowsdeclarealias{<new}{>new}{tonew}{tonew}

\makeatother

% tip length code
\pgfsetarrowoptions{latexnew}{-1pt}
\pgfsetarrowoptions{latex'new}{-1pt}
\pgfsetarrowoptions{onew}{-1pt}
\pgfsetarrowoptions{squarenew}{-1pt}
\pgfsetarrowoptions{stealthnew}{-1pt}
\pgfsetarrowoptions{tonew}{-1pt}
\pgfkeys{/tikz/.cd, arrowhead/.default=-1pt, arrowhead/.code={
  \pgfsetarrowoptions{latexnew}{#1},
  \pgfsetarrowoptions{latex'new}{#1},
  \pgfsetarrowoptions{onew}{#1},
  \pgfsetarrowoptions{squarenew}{#1},
  \pgfsetarrowoptions{stealthnew}{#1},
  \pgfsetarrowoptions{tonew}{#1},
}}

\usepackage{beamerposter} % Use the beamerposter package for laying out the poster

\usetheme{confposter} % Use the confposter theme supplied with this template

\setbeamercolor{block title}{fg=black,bg=white} % Colors of the block titles
\setbeamercolor{block body}{fg=black,bg=white} % Colors of the body of blocks
\definecolor{alertbgcolor}{rgb}{0.498,0.788,0.498}
\setbeamercolor{block alerted title}{fg=black,bg=alertbgcolor!70} % Colors of the highlighted block titles
\setbeamercolor{block alerted body}{fg=black,bg=white} % Colors of the body of highlighted blocks
% Many more colors are available for use in beamerthemeconfposter.sty

%-----------------------------------------------------------
% Define the column widths and overall poster size
% To set effective sepwid, onecolwid and twocolwid values, first choose how many columns you want and how much separation you want between columns
% In this template, the separation width chosen is 0.024 of the paper width and a 4-column layout
% onecolwid should therefore be (1-(# of columns+1)*sepwid)/# of columns e.g. (1-(4+1)*0.024)/4 = 0.22
% Set twocolwid to be (2*onecolwid)+sepwid = 0.464
% Set threecolwid to be (3*onecolwid)+2*sepwid = 0.708

\newlength{\panelspacing}
\newlength{\colwidth}
\newlength{\columnseparation}
\newlength{\sepwid}
\newlength{\onecolwid}
\newlength{\twocolwid}
\newlength{\threecolwid}
\setlength{\paperwidth}{36in} % A0 width: 46.8in
\setlength{\paperheight}{24in} % A0 height: 33.1in
\setlength{\colwidth}{0.31\paperwidth}
\setlength{\columnseparation}{0.0175\paperwidth}
\setlength{\sepwid}{0.024\paperwidth} % Separation width (white space) between columns
\setlength{\onecolwid}{0.22\paperwidth} % Width of one column
\setlength{\twocolwid}{0.464\paperwidth} % Width of two columns
\setlength{\threecolwid}{0.708\paperwidth} % Width of three columns
\setlength{\topmargin}{-0.5in} % Reduce the top margin size
%-----------------------------------------------------------

\usepackage{graphicx}  % Required for including images

\usepackage{booktabs} % Top and bottom rules for tables

\makeatletter
\addtobeamertemplate{block begin}{}{\setlength{\parindent}{2cm}\@afterindentfalse\@afterheading}
\makeatother

%----------------------------------------------------------------------------------------
%	TITLE SECTION 
%----------------------------------------------------------------------------------------

\title{\fontsize{50}{20} \selectfont Transfer Learning in Stacked LSTM's for Text Generation} % Poster title

\author{Alex Fischer} % Author(s)

\institute{University of Massachusetts, Amherst} % Institution(s)

%----------------------------------------------------------------------------------------

\begin{document}

\addtobeamertemplate{block end}{}{\vspace*{0ex}} % White space under blocks
\addtobeamertemplate{block alerted end}{}{\vspace*{2ex}} % White space under highlighted (alert) blocks

\setlength{\belowcaptionskip}{0ex} % White space under figures
\setlength\belowdisplayshortskip{2ex} % White space under equations

\begin{frame}[t] % The whole poster is enclosed in one beamer frame

\begin{columns}[t] % The whole poster consists of three major columns, the second of which is split into two columns twice - the [t] option aligns each column's content to the top

\begin{column}{\columnseparation}\end{column} % Empty spacer column

\begin{column}{\colwidth} % The first column

%----------------------------------------------------------------------------------------
%	INTRODUCTION
%----------------------------------------------------------------------------------------

\begin{block}{Stacked LSTM's}

Nerve regeneration is an active area of research in the medical community. In order to test the effects of various treatments on regeneration, researchers at the Miami Project to Cure Paralysis use mice optic nerves as a model for neuron growth in general. After crushing mice optic nerves, subjecting them to various treatments, and letting them grow, they obtain 3D images of the optic nerves, specifically of the axons within the nerves. They then seek to analyze the axons, as axons in normal nerves grow in straight paths, but injured nerves feature, wavy, deformed axons.
%Professor Vance Lemmon, as part of the Miami Project to Cure Paralysis, is running experiments about regeneration in injured mice optic nerves. As part of the experiment, he obtains 3D images of mice optic nerves in order to analyze the growth of the axons within. Axons in uninjured nerves grow in regular, straight paths, whereas axons in injured nerves exhibit highly deformed paths and rarely grow past the injury site of the nerve. Professor Lemmon analyzes the paths that the axons take in order to evaluate the effect of various treatments applied to mice with injured optic nerves.

To analyze the axons, researchers must manually trace the axons, a highly tedious and time-consuming task. Factors such as noise in the images and intersecting axons make this task exceptionally difficult for both humans and computers. The purpose of this project is to develop software that automates or substantially reduces the amount of human involvement in the process of tracing the paths of the axons. Subsequently, we would use those tracings to analyze the paths the axons take, identifying features such as branches and quantifying the deformity of the axons. To do this, we would develop a pipeline that starts with a 3D image and ends with analysis of paths of axons in the iamge, with several intermediate steps.
%In order to analyse the axons in the optic nerves, members of Professor Lemmon's lab must manually trace the axons in the resultant 3D images, a highly tedious and labor-intensive process that takes several days per image and must be repeated for many images. The purpose of this project is to develop software that automates or at least substantially reduces the amount of human interaction involved in tracing the paths of the axons. Subsequently, we would use those tracings to analyze the paths the axons take, identifying features such as branches and quantifying the extent to which the paths of the axons are deformed. To do this, we would develop a pipeline that starts with a 3D image and ends with analysis or paths of axons in the iamge, with several intermediate steps.

\begin{figure}

\tikzstyle{block} = [rectangle, draw, %fill=blue!20,
    text width=5em, text centered, rounded corners, minimum height=4em]
\tikzstyle{line} = [draw, very thick, color=black!50, -latexnew, arrowhead=1cm]

\begin{tikzpicture}[scale=1, node distance = 15cm, auto]
    % Place nodes
    \node [block] (original) {3D image};
    \node [block, right of=original] (denoised) {denoised image};
    \node [block, right of=denoised] (segment) {segmented image};
    \node [block, below of=segment, node distance=7cm] (skeleton) {3D skeleton};
    \node [block, left of=skeleton] (graph) {graph};
    \node [block, left of=graph] (analyzed) {analyzed axons};
    % Draw edges
    \path [line] (original) -- node [color=black] {denoising} (denoised);
    \path [line] (denoised) -- node [color=black] {segmentation} (segment);
    \path [line] (segment) -- node [color=black, left] {skeletonization} (skeleton);
    \path [line] (skeleton) -- node [color=black] {graph building} (graph);
    \path [line] (graph) -- node [color=black] {analysis} (analyzed);
\end{tikzpicture}

\caption{The proposed pipeline from a 3D iamge to analyzed axon paths.}

\end{figure}

\end{block}

\begin{block}{Denoising}
\begin{itemize}
	\item Images are noisy due to imperfect optics and uneven diffusion of marker in axons.
	\item Simple denoising algorithms blur image, removing small, thin features such as axons.
	\item Nonlocal means denoising \cite{nonlocalmeans} is a more sophisticated denoising algorithm that preserves small features by doing global, not local averaging of voxel values.
\end{itemize}

\end{block}

\end{column} % End of the first column

\begin{column}{\columnseparation}\end{column} % Empty spacer column

\begin{column}{\colwidth} % The second column

\begin{block}{Segmentation}

\begin{itemize}
	\item We tried several existing segmentation algorithms.
	\begin{itemize}	
		\item Thresholding based on intensity value picks up too much background noise while not having axons be sufficiently connected.
		\item Graphcut \cite{graphcut} improved upon thresholding by picking up less noise and having axons be more connected, but was not satisfactory and was much slower than thresholding.
	\end{itemize}
	\item Standard segmentation algorithms fail because they fail to take into account information about the structures they segment. Axons are thin and elongated, and humans use this information when manually segmenting.
	\item We devised our own segmentation algorithm that takes advantage of this information.
\end{itemize}

\begin{alertblock}{Enhancing Thin Structures}
\begin{itemize}
	\item Examine the volume in a series of small windows: large enough to capture axons and background, but small enough so that axons are locally fairly linear.
	\item Fit a multivariate normal distribution to the window and compute its mean and variances.
	\item Compute the ratio of the variances along the first and second principal axes of the distribution. A high ratio means the window contains a thin, elongated structure in the direction of the first principal axis.
	\item Increase the intensity of voxels near the mean of the distribution based on the ratio of variances. If there is a higher ratio, the window is more likely to contain an axon and thus should be enhanced more.
\end{itemize}

\end{alertblock}
\vspace{-1cm}
\begin{itemize}
	\item Thresholding the enhanced image resulted in less background noise being detected along with more plentiful axons that were better connected.
\end{itemize}

\end{block}
\begin{block}{Skeletonization}

\begin{itemize}
	\item A skeleton of a 3D object is a topological representation of its features consisting of one voxel wide segments.
	\item Skeletons contain all information about axon paths while being easier to work with than a segmented volume.
	\item We used an existing skeletonization algorithm \cite{skeletonization} that builds a skeleton by progressively thinning a shape until all features are one voxel thick.
\end{itemize}

\end{block}

\end{column} % End of the second column

\begin{column}{\columnseparation}\end{column} % Empty spacer column

\begin{column}{\colwidth} % The third column

\begin{block}{Graph Building}

\begin{itemize}
	\item We built a graph to represent the paths and intersections between paths of the skeleton.
	\item We removed imperfections of the skeletonization algorithm by filtering out endpoint paths with less than a certain length and by removing vertices with only two edges.
\end{itemize}

\end{block}

\begin{block}{Future Work: Path Detection and Analysis}

\begin{itemize}
	\item Axons often grow alongside each other due to cell adhesion, and frequently intersect.
	\item We need to distinguish those intersections from true branches: when one axon splits into two axons that grow in different directions. To do this, we will examine the directions that segments go in at vertices.
	\item We also need to quantify the tortuosity of the axon paths. There are several measures that could perform this.
		\begin{itemize}
			\item Ratio of path length to distance between endpoints.
			\item Integrate some approximation of local curvature along the path.
			\item Compute fractal measure of path by measuring length using segments of various length.
		\end{itemize}
\end{itemize}

\end{block}

%----------------------------------------------------------------------------------------
%	ACKNOWLEDGEMENTS
%----------------------------------------------------------------------------------------

%\setbeamercolor{block title}{fg=red,bg=white} % Change the block title color

\begin{block}{Acknowledgements}

I would like to thank Joseph Masterjohn for the assistance that he provided during this project. I would like to thank Eric Bray, Kevin Park and Pantelis Tsoulfas for sharing unpublished data for this project. I would also like to thank Professor Burton Rosenberg, and all of the others who coordinated this Research Experience for Undergraduates program, for doing so. I would like to thank the University of Miami's Department of Computer Science and their Center for Computational Science for running this program. I would like to thank the Miami Project to Cure Paralysis. Lastly, I would like to thank the National Science Foundation for generously funding this program.

\end{block}

%----------------------------------------------------------------------------------------
%	REFERENCES
%----------------------------------------------------------------------------------------

\begin{block}{References}

\nocite{*} % Insert publications even if they are not cited in the poster
\begin{thebibliography}{9}
\bibitem{graphcut} Boykov, Yuri Y., and M-P. Jolly. "Interactive graph cuts for optimal boundary \& region segmentation of objects in ND images." In \textit{Computer Vision, 2001. ICCV 2001. Proceedings. Eighth IEEE International Conference on}, vol. 1, pp. 105-112. IEEE, 2001.

\bibitem{nonlocalmeans} Buades, Antoni, Bartomeu Coll, and J-M. Morel. "A non-local algorithm for image denoising." In \textit{Computer Vision and Pattern Recognition, 2005. CVPR 2005. IEEE Computer Society Conference on}, vol. 2, pp. 60-65. IEEE, 2005.

\bibitem{regionpushrelabel} Delong, Andrew, and Yuri Boykov. "A scalable graph-cut algorithm for ND grids." In \textit{Computer Vision and Pattern Recognition, 2008. CVPR 2008. IEEE Conference on}, pp. 1-8. IEEE, 2008.

\bibitem{skeletonization} Lee, Ta-Chih, Rangasami L. Kashyap, and Chong-Nam Chu. "Building skeleton models via 3-D medial surface axis thinning algorithms." \textit{CVGIP: Graphical Models and Image Processing} 56, no. 6 (1994): 462-478.

\end{thebibliography}
\end{block}

%\begin{center}
%\begin{tabular}{ccc}
%\includegraphics[width=0.4\linewidth]{ccs.png} & \hfill & \includegraphics[width=0.4\linewidth]{cs.png}
%\end{tabular}
%\end{center}

%----------------------------------------------------------------------------------------

\end{column} % End of the third column

\end{columns} % End of all the columns in the poster

\end{frame} % End of the enclosing frame

\end{document}
